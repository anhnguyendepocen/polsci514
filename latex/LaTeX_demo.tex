\documentclass[11pt]{article}

% package
\usepackage{amsmath,amssymb,amsthm}

% margin
\addtolength{\evensidemargin}{-.5in}
\addtolength{\oddsidemargin}{-.5in}
\addtolength{\textwidth}{0.8in}
\addtolength{\textheight}{0.8in}
\addtolength{\topmargin}{-.4in}

% custom definitions
\newtheoremstyle{quest}{\topsep}{\topsep}{}{}{\bfseries}{}{ }{\thmname{#1}\thmnote{ #3}.}
\theoremstyle{quest}
\newtheorem*{definition}{Definition}
\newtheorem*{theorem}{Theorem}
\newtheorem*{question}{Question}
\newtheorem*{exercise}{Exercise}
\newtheorem*{challengeproblem}{Challenge Problem}
\newcommand{\name}{

%% put your name here, so we know who to give credit to %%
[Your Name Here]
}
\newcommand{\hw}{
%% and which homework assignment is it? put the correct number below        
1
}

\title{\vspace{-50pt}
\Huge \name
\\\vspace{20pt}
\huge POLSCI 514\hfill Problem Set \hw}
\author{}
\date{}
\pagestyle{myheadings}
\markright{\name\hfill Problem Set \hw\qquad\hfill}

%% If you want to define a new command, you can do it like this:
%% The followins is the useful commands for math and probability. 
\newcommand{\R}{\mathbb{R}}
\newcommand{\E}{\mathbb{E}}
\newcommand{\V}{\mathbb{V}}



\begin{document}
\maketitle



\section{Writing in math mode}
The most important thing is how to write down mathematical expressions in \LaTeX.
There are several ways to write equations. 

\subsection{Inline equation}
The first way is to enclose equations in dollar signs: $x+y=a^2+b^2$.
This is useful when you want to embed a simple equation in your text.

\subsection{One equation in a new line}
If you want to use new line for equations, there are two ways:

\[x+y=a^2+b^2\]

or

$$
x+y=a^2+b^2
$$

\subsection{Many equations in new lines}
If you have a chain of deductions, you can line them up like this.
The \& tells it where to line up---usually you will want it right before the equals sign.
You should have an 'and' symbol, \&, in each line.
The double-backslash, $\backslash\backslash$, means indentation. 
\begin{equation}
\begin{split}
  2x &= 2y-10 \\
  x &=y -5   \\
\end{split}
\end{equation}


If you don't want to enumerate the equations, add an asterisk:
\begin{equation}
\begin{split}
  2x &= 2y-10 \\
  x &=y -5  
\end{split}
\end{equation}

You might want to include words if they are short:
\begin{equation}
\begin{split}
  2x &= 2y-10 \quad \text{bla bla bla}\\
  x &=y -5  
\end{split}
\end{equation}


\subsection{Symbols}

There are some symbols that often come up in the homework. \\
\begin{itemize}
    \item $\mathbb{E}$: Expectation  
    \item $\mathbb{V}$: Variance
    \item $\mathbb{P}$: Probability measure
    \item $\mathbb{R}$: Set of real numbers
\end{itemize}
This is new area.

In this template, you can simply type $\E$, $\V$ and $\R$ (but not P becasue a backslash and P is already defined as $\P$.) 
See the header for the alphabets with this functionality. 

You can write Greek letters in the way they are read. $\alpha$, $\beta$, $\gamma$, 
$\lambda$, $\omega$. Capitalize the first letter to get the capital letters. $\Omega$

Other useful symbols are like this. 
\begin{itemize}
    \item $\cup$: union
    \item $\cap$: intersection
    \item $a \in A$ : $a$ is an element of $A$
    \item $A = \{1, 2, 3\}$: specify a set by its elements. 
    % For { }, you have to escape by a backslash. i.e. \{ \}
    \item $\mathcal{F}$: set of events
    \item $\sum_{i=1}^n$: summation 
    % if you have more than one letter in sub/superscript, use {}.
\end{itemize}

Finally, you can write subscripts $X_i$ and superscripts $X^2$
Use curly parentheses if you want to use multiple characters in the subscripts and superscripts such as $X_{ij}$ and $X_{20}$. 

\newpage
\section{Inserting figures}

\newpage 
\section{Inserting tables}

An example of displaying summary statistics from a survey.


\begin{table}[h!]
\centering
\begin{tabular}{ll|rr}
% NOTE: {l: left, r: right, c: center} will specify the location of text within a cell.
\hline
\hline
\textbf{Variable} & \textbf{Levels} & $\mathbf{n}$ & $\mathbf{\%}$ \\ 
\hline
Gender & Male & 1531 & 52.8 \\ 
   & Female & 1411 & 47.8 \\ 
   & Other & 1 & 0.0 \\ 
   & NA & 11 & 0.4\\
\hline
\hline
Age & 19-30 & 474 & 16.0 \\ 
   & 30-40 & 597 & 20.2 \\ 
   & 40-50 & 704 & 23.8 \\ 
   & 50-60 & 566 & 19.2 \\ 
   & 60-79 & 613 & 20.8 \\ 
\hline
\hline
Education & College & 1688 & 57.1 \\ 
   & Not College & 1230 & 41.6 \\ 
   & NA & 36 & 1.2 \\ 
\hline
\hline
\end{tabular}
\label{tab:sum}
\caption{Table of summary statistics about the respondents. 
  }
\end{table}


\newpage
\section{Utility features for problem sets}
Some utility functions are defined in the header of this file. 
The following is some examples.

\begin{question}[1]
Here is your question.
\end{question}
\begin{proof}
Here is my proof.
\end{proof}

\begin{question}[2b]
Another question.
\end{question}
\begin{proof}
Another proof.
\end{proof}

There are similar environments for Theorems.

\begin{theorem}[3]
  Some theorem.
\end{theorem}
\begin{proof}
  Insert your proof here.
\end{proof}

%%%% don't delete the last line!
\end{document}
